\IEEEPARstart{N}{owadays} searches look at facial features detection, and facial expression recognition. Indeed, for analyse a face there is several methods. Principal Component Analysis (PCA), Linear Discriminant Analysis (LDA), and Locality Preserving Projection (LPP) are most used methods \cite{UsingGraphModelforFaceAnalysis}.

\subsection{Typical methods}
	In this part, we will quickly present each typical methods employed for analyse a face\cite{UsingGraphModelforFaceAnalysis}. These approchs are not used alone, they are often combinated.\\
	
	
	\subsubsection{Principal Component Analysis} \leavevmode\par
	PCA is a useful statistical technique that has found application in fields such as face recognition and image compression, and is a common technique for finding patterns in data of high dimension. For use PCA, we need to have several images available. PCA represent images such as this :
		
		For an image I (nxn), there is $n^{2}$ pixels.
		
		For k images PCA extract $n^{2}$ k-dimensional vectors.
		
		After that PCA reduce vector space dimension. The result of the five step of PCA give us a simple way to know if two images are neighbor.

		Compress each of them with PCA and compare.
		
	\subsubsection{Linear Discriminant Analysis}

	\subsubsection{Locality Preserving Projection}

\subsection{Methods using based on typical methods}
	\subsubsection{PCA Extensions\\}
		A searchers team worked on a PCA extension, a Two-Dimensional PCA\cite{Two-Dimensional-PCA}. In this approch, they use PCA but dsfdsfsdf
	
	\subsubsection{Active Shape Models}
		
	\subsubsection{Graph Model for Face Analysis}


